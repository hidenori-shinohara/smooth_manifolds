\subsection{Topological Manifolds}

\begin{defn}
  A topological $n$-manifold is a Hausdorff, second-countable topological space each point of which has a neighborhood that is homeomorphic to an open subset $\mathbb{R}^n$.
\end{defn}

\begin{defn}
  Let $M$ be a topological $n$-manifold.
  Let $U$ be an open subset of $M$, $\hat{U}$ be an open subset of $\mathbb{R}^n$, $\phi: U \rightarrow \hat{U}$ be a homeomorphism.
  \begin{itemize}
    \item
      The pair $(U, \phi)$ is called a coordinate chart or a chart.
    \item
      $U$ is called a coordinate domain or a coordinate neighborhood.
    \item
      If $\phi(U)$ is an open ball in $\mathbb{R}^n$, $U$ is called a coordinate ball.
    \item
      If $\phi(U)$ is an open cube in $\mathbb{R}^n$, $U$ is called a coordinate cube and $\phi$ is called a coordinate map.
  \end{itemize}
\end{defn}
