\subsection{Topological Manifolds}

\begin{defn}[Topological Manifold]
  A \textit{topological $n$-manifold} is a Hausdorff, second-countable topological space each point of which has a neighborhood that is homeomorphic to an open subset $\mathbb{R}^n$.
\end{defn}

\begin{defn}[Coordinates]
  Let $M$ be a topological $n$-manifold.
  Let $U$ be an open subset of $M$, $\hat{U}$ be an open subset of $\mathbb{R}^n$, $\phi: U \rightarrow \hat{U}$ be a homeomorphism.
  \begin{itemize}
    \item
      The pair $(U, \phi)$ is called a \textit{coordinate chart} or a \textit{chart}.
    \item
      $U$ is called a \textit{coordinate domain} or a \textit{coordinate neighborhood} and $\phi$ is called a \textit{coordinate map}.
    \item
      If $\phi(U)$ is an open ball in $\mathbb{R}^n$, $U$ is called a \textit{coordinate ball}.
    \item
      If $\phi(U)$ is an open cube in $\mathbb{R}^n$, $U$ is called a \textit{coordinate cube}.
  \end{itemize}
\end{defn}

\begin{defn}[Atlas]
  Let $M$ be a topological $n$-manifold.
  An \textit{atlas for $M$} is a collection of charts $(U_{\alpha}, \phi_{\alpha})$ such that $M = \bigcup_{\alpha} U_{\alpha}$.
\end{defn}

\begin{defn}[Transition Map]
  Let $M$ be a topological $n$-manifold and $(U, \phi), (V, \psi)$ be coordinate charts such that $U \cap V \ne \emptyset$.
  $\psi \circ \phi^{-1}: \phi(U \cap V) \mapsto \psi(U \cap V)$ is called a \textit{transition map} from $\phi$ to $\psi$.
\end{defn}

\begin{defn}[Closed Upper Half-Space]
  $\mathbb{H}^n = \{ (x^1, \cdots, x^n) \in \mathbb{R}^n \mid x^n \geq 0 \}$, and $\partial \mathbb{H}^n = \{ (x^1, \cdots, x^n) \in \mathbb{R}^n \mid x^n = 0 \}$.
\end{defn}

\begin{defn}[Manifold With Boundary]
  Let $M$ be a second-countable Hausdorff space and fix $n$.
  Suppose that for every $p \in M$, one of the following conditions is satisfied:
  \begin{enumerate}
    \item
      There exists a neighborhood $U$ of $p$ and a homeomorphism $\phi:U \rightarrow \hat{U}$ where $\hat{U}$ is an open subset of $\mathbb{R}^n$.
      $p$ is called an \textit{interior point} and $(U, \phi)$ is called an \textit{interior chart}.
    \item
      There exists a neighborhood $U$ of $p$ and a homeomorphism $\phi:U \rightarrow \hat{U}$ where $\hat{U}$ is an open subset of $\mathbb{H}^n$ with $\hat{U} \cap \partial \mathbb{H}^n \ne \emptyset$.
      $p$ is called a \textit{boundary point} and $(U, \phi)$ is called a \textit{boundary chart}.
  \end{enumerate}
  Then $M$ is called an \textit{$n$-dimensional topological manifold with boundary}.
  Note that every topological manifold is a topological manifold with boundary.
\end{defn}


\subsection{Smooth Manifolds}

\begin{defn}[Smoothly Compatible]
  Let $M$ be a topological $n$-manifold.
  Two coordinate charts $(U, \phi), (V, \psi)$ are called \textit{smoothly compatible} if $U \cap V = \emptyset$ or the transition map $\psi \circ \phi^{-1}$ is a diffeomorphism.
\end{defn}

\begin{defn}[Smooth Atlas]
  Let $M$ be a topological $n$-manifold.
  A \textit{smooth atlas} is an atlas $\mathcal{A}$ such that any two charts in $\mathcal{A}$ are smoothly compatible with each other.
\end{defn}

\begin{defn}[Smooth Structure]
  If $M$ is a topological $n$-manifold, an atlas $\mathcal{A}$ that is not properly contained in any larger smooth atlas is called \textit{maximal} or a \textit{smooth structure on $M$}
\end{defn}

\begin{defn}[Smooth Manifold]
  A \textit{smooth manifold} is a topological manifold equipped with a smooth structure.
\end{defn}

\begin{defn}
  Suppose $(M, \mathcal{A})$ is a smooth manifold.
  \begin{itemize}
    \item
      Any chart $(U, \phi) \in \mathcal{A}$ is called a \textit{smooth chart}.
    \item
      Given a smooth chart $(U, \phi)$, $U$ is called a \text{smooth coordinate domain} and $\phi$ is called a \textit{smooth coordinate map}.
    \item
      Given a smooth chart $(U, \phi)$, $U$ is called a \textit{smooth coordinate ball} if it is a coordinate ball.
  \end{itemize}
\end{defn}

\begin{rem}
  One must define a smooth structure on a topological manifold before talking about a smooth chart.
\end{rem}

\begin{defn}[Smooth Maps]
  Let $M, N$ be smooth manifolds and $F: M \rightarrow N$ be a map.
  $F$ is a \textit{smooth map} if for every $p \in M$, there exist smooth charts $(U, \phi)$ containing $p$ and $(V, \psi)$ containing $F(p)$ such that
  \begin{itemize}
    \item
      $F(U) \subset V$;
    \item
      $\psi \circ F \circ \phi^{-1}: \phi(U) \rightarrow \psi(V)$ is smooth.
  \end{itemize}
\end{defn}

\subsection{Tangent Vectors}

\begin{defn}[Derivation]
  Let $M$ be a smooth manifold with or without boundary.
  A derivation at $p \in M$ is a linear map $v: C^{\infty}(M) \rightarrow \mathbb{R}$ such that 

  \begin{align*}
    v(fg) = f(p)vg + g(p)vf
  \end{align*}

  for all $f, g \in C^{\infty}(M)$.
\end{defn}

\begin{defn}[Tangent Space]
  The tangent space $T_pM$ to $M$ at $p$ is the vector space of all derivations of $C^{\infty}(M)$ at $p$.
\end{defn}

\begin{defn}[Differential]
  $M, N$ are smooth manifolds with or without boundary, and $F: M \rightarrow N$ is a smooth map.
  The \textit{differential of $F$ at $p$} is the linear map $dF_p: T_pM \rightarrow T_{F(p)}N$ defined by 
  \begin{align*}
    dF_p(v) := f \mapsto v(f \circ F)
  \end{align*}

  Equivalently, $\forall v \in T_pM, \forall f \in C^{\infty}(N), dF_p(v)(f) = v(f \circ F)$.
\end{defn}
