\subsection{Topological Manifolds}

\begin{defn}[Topological Manifold]
  A \textit{topological $n$-manifold} is a Hausdorff, second-countable topological space each point of which has a neighborhood that is homeomorphic to an open subset $\mathbb{R}^n$.
\end{defn}

\begin{defn}[Coordinates]
  Let $M$ be a topological $n$-manifold.
  Let $U$ be an open subset of $M$, $\hat{U}$ be an open subset of $\mathbb{R}^n$, $\phi: U \rightarrow \hat{U}$ be a homeomorphism.
  \begin{itemize}
    \item
      The pair $(U, \phi)$ is called a \textit{coordinate chart} or a \textit{chart}.
    \item
      $U$ is called a \textit{coordinate domain} or a \textit{coordinate neighborhood} and $\phi$ is called a \textit{coordinate map}.
    \item
      If $\phi(U)$ is an open ball in $\mathbb{R}^n$, $U$ is called a \textit{coordinate ball}.
    \item
      If $\phi(U)$ is an open cube in $\mathbb{R}^n$, $U$ is called a \textit{coordinate cube}.
    \item
      The coordinate functions of $\phi$ are often denoted as $(x^1, \cdots, x^n)$.
      Thus a chart is sometimes denoted by $(U, (x^1, \cdots, x^n))$ or $(U, (x^i))$.
  \end{itemize}
\end{defn}

\begin{defn}[Atlas]
  Let $M$ be a topological $n$-manifold.
  An \textit{atlas for $M$} is a collection of charts $(U_{\alpha}, \phi_{\alpha})$ such that $M = \bigcup_{\alpha} U_{\alpha}$.
\end{defn}

\begin{defn}[Transition Map]
  Let $M$ be a topological $n$-manifold and $(U, \phi), (V, \psi)$ be coordinate charts such that $U \cap V \ne \emptyset$.
  $\psi \circ \phi^{-1}: \phi(U \cap V) \mapsto \psi(U \cap V)$ is called a \textit{transition map} from $\phi$ to $\psi$.
\end{defn}

\begin{defn}[Closed Upper Half-Space]
  $\mathbb{H}^n = \{ (x^1, \cdots, x^n) \in \mathbb{R}^n \mid x^n \geq 0 \}$, and $\partial \mathbb{H}^n = \{ (x^1, \cdots, x^n) \in \mathbb{R}^n \mid x^n = 0 \}$.
\end{defn}

\begin{defn}[Manifold With Boundary]
  Let $M$ be a second-countable Hausdorff space and fix $n$.
  Suppose that for every $p \in M$, one of the following conditions is satisfied:
  \begin{enumerate}
    \item
      There exists a neighborhood $U$ of $p$ and a homeomorphism $\phi:U \rightarrow \hat{U}$ where $\hat{U}$ is an open subset of $\mathbb{R}^n$.
      $p$ is called an \textit{interior point} and $(U, \phi)$ is called an \textit{interior chart}.
    \item
      There exists a neighborhood $U$ of $p$ and a homeomorphism $\phi:U \rightarrow \hat{U}$ where $\hat{U}$ is an open subset of $\mathbb{H}^n$ with $\phi(p) \in \partial \mathbb{H}^n$.
      $p$ is called a \textit{boundary point}.
  \end{enumerate}
  Then $M$ is called an \textit{$n$-dimensional topological manifold with boundary}.
  Note that every topological manifold is a topological manifold with boundary.
\end{defn}

\begin{defn}[Support]
  If $f$ is any real-valued or vector-valued function on a topological space $M$, the \textit{support of $f$}, denoted by $\supp f$, is the closure of the set of points where $f$ is nonzero:
  \begin{align*}
    \supp f = \overline{\{p \in M : f(p) \ne 0 \}}.
  \end{align*}
\end{defn}

\begin{defn}[Bump Function]
  If $M$ is a topological space, $A \subset M$ is a closed subset, and $U \subset M$ is an open subset containing $A$, a continuous function $\psi: M \rightarrow \mathbb{R}$ is called a \textit{bump function for $A$ supported in $U$} if $0 \leq \psi \leq 1$ on $M$, $\psi \equiv 1$ on $A$, and $\supp \psi \subset U$.
\end{defn}

\subsection{Smooth Manifolds}

\begin{defn}[Smoothly Compatible]
  Let $M$ be a topological $n$-manifold.
  Two coordinate charts $(U, \phi), (V, \psi)$ are called \textit{smoothly compatible} if $U \cap V = \emptyset$ or the transition map $\psi \circ \phi^{-1}$ is a diffeomorphism.
\end{defn}

\begin{defn}[Smooth Atlas]
  Let $M$ be a topological $n$-manifold.
  A \textit{smooth atlas} is an atlas $\mathcal{A}$ such that any two charts in $\mathcal{A}$ are smoothly compatible with each other.
\end{defn}

\begin{defn}[Smooth Structure]
  If $M$ is a topological $n$-manifold, an atlas $\mathcal{A}$ that is not properly contained in any larger smooth atlas is called \textit{maximal} or a \textit{smooth structure on $M$}
\end{defn}

\begin{defn}[Smooth Manifold]
  A \textit{smooth manifold} is a topological manifold equipped with a smooth structure.
\end{defn}

\begin{defn}
  Suppose $(M, \mathcal{A})$ is a smooth manifold.
  \begin{itemize}
    \item
      Any chart $(U, \phi) \in \mathcal{A}$ is called a \textit{smooth chart}.
    \item
      Given a smooth chart $(U, \phi)$, $U$ is called a \text{smooth coordinate domain} and $\phi$ is called a \textit{smooth coordinate map}.
    \item
      Given a smooth chart $(U, \phi)$, $U$ is called a \textit{smooth coordinate ball} if it is a coordinate ball.
  \end{itemize}
\end{defn}

\begin{rem}
  One must define a smooth structure on a topological manifold before talking about a smooth chart.
\end{rem}

\begin{defn}[Smooth Maps]
  Let $M, N$ be smooth manifolds with or without boundary and $F: M \rightarrow N$ be a map.
  $F$ is a \textit{smooth map} if for every $p \in M$, there exist smooth charts $(U, \phi)$ containing $p$ and $(V, \psi)$ containing $F(p)$ such that
  \begin{itemize}
    \item
      $F(U) \subset V$;
    \item
      $\psi \circ F \circ \phi^{-1}: \phi(U) \rightarrow \psi(V)$ is smooth.
  \end{itemize}
\end{defn}

\begin{defn}[Coordinate Representatin of a Smooth Map]
  Let $(M, \mathcal{A}_M)$ and $(N, \mathcal{A}_N)$ be smooth manifolds.
  Let $F: M \rightarrow N$ be a smooth map and $(U, \phi) \in \mathcal{A}_M$ and $(V, \psi) \in \mathcal{A}_N$ be given.
  Then $\hat{F} = \psi \circ F \circ \phi^{-1}$ is called the coordinate representation of $F$ with respect to $(U, \phi)$ and $(V, \psi)$.
\end{defn}

\begin{defn}[Diffeomorphism]
  Let $M, N$ be smooth manifolds with or without boundary.
  A diffeomorphism is a smooth map $F: M \rightarrow N$ with a smooth inverse.
\end{defn}

\begin{defn}[Smooth on a subset]
  Let $M, N$ be smooth manifolds with or without boundary and $A \subset M$ be an arbitrary subset.
  A map $F: A \rightarrow N$ is said to be \textit{smooth on $A$} if every $p \in A$ has an open neighborhood $W \subset M$ such that there exists a smooth map $\tilde{F}:W \rightarrow N$ with $\tilde{F}_{W \cap A} = F$.
\end{defn} 

\subsection{Tangent Vectors}

\begin{defn}[Derivation]
  Let $M$ be a smooth manifold with or without boundary.
  A derivation at $p \in M$ is a linear map $v: C^{\infty}(M) \rightarrow \mathbb{R}$ such that 

  \begin{align*}
    v(fg) = f(p)vg + g(p)vf
  \end{align*}
  for all $f, g \in C^{\infty}(M)$.

  This corresponds to ``arrows that are tangent to $M$ and whose basepoints are attached to $M$ at $p$" even though it may not be easy to see that from this definition.
\end{defn}

\begin{defn}[Tangent Space]
  The tangent space $T_pM$ to $M$ at $p$ is the vector space of all derivations of $C^{\infty}(M)$ at $p$.
\end{defn}

\begin{center}
  \begin{tabular}{ | l | l | }
    \hline
    Derivation of $C^{\infty}(M)$ & Geometric tangent vector on $M$ \\ \hline
    Differential of a smooth map between manifolds & Total derivative of a map between Euclidean spaces \\ \hline
  \end{tabular}
\end{center}

\begin{defn}[Differential]
  $M, N$ are smooth manifolds with or without boundary, and $F: M \rightarrow N$ is a smooth map.
  The \textit{differential of $F$ at $p$} is the linear map $dF_p: T_pM \rightarrow T_{F(p)}N$ defined by 
  \begin{align*}
    dF_p(v) := f \mapsto v(f \circ F)
  \end{align*}
  Equivalently, $\forall v \in T_pM, \forall f \in C^{\infty}(N), dF_p(v)(f) = v(f \circ F)$.
  This corresponds to ``the directional derivative of $F$ at $p$ in the direction of the arrow $v$."
\end{defn}

\begin{defn}[Coordinate Vectors]
  Let $(M, \mathcal{A})$ be a smooth manifold without boundary.
  Let $p \in M$ and choose a chart $(U, \phi) \in \mathcal{A}$ such that $p \in U$.
  Then the \textit{coordinate vectors at $p$}, denoted by $\frac{\partial}{\partial x^i}\vert_p$, are derivations $C^{\infty}(U) \rightarrow \mathbb{R}$ such that

  \begin{align*}
    \frac{\partial}{\partial x^i}\Big{\vert}_p := f \mapsto \frac{\partial}{\partial x^i}\Big{\vert}_{\phi(p)}(f \circ \phi^{-1}).
  \end{align*}
\end{defn}

\begin{defn}[Tangent Bundle]
  Let $M$ be a smooth manifold with or without boundary.
  The tangent bundle of $M$, denoted by $TM$, is the disjoint union $\coprod_{p \in M} T_pM$.
\end{defn}

\begin{defn}[Projection Map]
  Let $M$ be a smooth manifold with or without boundary.
  The projection map $\pi: TM \rightarrow M$ is the map defined by $(p, v) \mapsto p$.
\end{defn}

\begin{defn}[Curve]
  If $M$ is a manifold with or without boundary, we define a \textit{curve in $M$} to be a continuous map $\gamma: J \rightarrow M$ where $J \subset \mathbb{R}$ is an interval.
\end{defn}

\begin{defn}[Velocity of a curve]
  Let $\gamma: J \rightarrow M$ and $t_0 \in J$ be given.
  The \textit{velocity of $\gamma$ at $t_0$}, denoted by $\gamma'(t_0)$ is the vector
  \begin{align*}
    \gamma'(t_0) = d\gamma(\frac{d}{dt}\Big\vert_{t_0}) \in T_{\gamma(t_0)}M,
  \end{align*}
  where $d/dt\vert_{t_0}$ is the standard coordinate basis vector in $T_{t_0}\mathbb{R}$.
\end{defn}

\subsection{Submersions, Immersions, and Embeddings}

\begin{defn}[Rank]
  Let $M, N$ be smooth manifolds with or without boundary and let $F: M \rightarrow N$ be a smooth map.
  Then the rank of $F$ at $p \in M$ is:
  \begin{itemize}
    \item
      The rank of the linear map $dF_p: T_pM \rightarrow T_{F(p)}N$.
    \item
      The dimension of the subspace $dF_p(T_pM)$ in the vector space $T_{F(P)}N$.
  \end{itemize}
  It is easy to see that the two definitions above are always equivalent.
\end{defn}

\begin{defn}[Submersions and Immersions]
  Let $M, N$ be smooth manifolds with or without boundary and let $F: M \rightarrow N$ be a smooth map.
  \begin{itemize}
    \item
      If $F$ has the same rank at every point $p \in M$, then $F$ is said to have \textit{constant rank}, and the rank is denoted by $\rank F$.
    \item
      If the rank of $F$ at $p \in M$ is equal to $\max \{ \dim M, \dim N \}$, then $F$ is said to have \textit{full rank at $p$}.
    \item
      If $F$ has full rank everywhere, then $F$ is said to have \textit{full rank}.
    \item
      If $F$ has constant rank and $\rank F = \dim N$, $F$ is called a \textit{smooth submersion}.
    \item
      If $F$ has constant rank and $\rank F = \dim M$, $F$ is called a \textit{smooth immersion}.
  \end{itemize}
\end{defn}

