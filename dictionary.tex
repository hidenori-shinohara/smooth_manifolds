\subsection{Topological Manifolds}

\begin{defn}[Topological Manifold]
  A \textit{topological $n$-manifold} is a Hausdorff, second-countable topological space each point of which has a neighborhood that is homeomorphic to an open subset $\mathbb{R}^n$.
\end{defn}

\begin{defn}[Coordinates]
  Let $M$ be a topological $n$-manifold.
  Let $U$ be an open subset of $M$, $\hat{U}$ be an open subset of $\mathbb{R}^n$, $\phi: U \rightarrow \hat{U}$ be a homeomorphism.
  \begin{itemize}
    \item
      The pair $(U, \phi)$ is called a \texit{coordinate chart} or a \textit{chart}.
    \item
      $U$ is called a \textit{coordinate domain} or a \textit{coordinate neighborhood} and $\phi$ is called a \textit{coordinate map}.
    \item
      If $\phi(U)$ is an open ball in $\mathbb{R}^n$, $U$ is called a \textit{coordinate ball}.
    \item
      If $\phi(U)$ is an open cube in $\mathbb{R}^n$, $U$ is called a \textit{coordinate cube}.
  \end{itemize}
\end{defn}

\begin{defn}[Atlas]
  Let $M$ be a topological $n$-manifold.
  An \textit{atlas for $M$} is a collection of charts $(U_{\alpha}, \phi_{\alpha})$ such that $M = \bigcup_{\alpha} U_{\alpha}$.
\end{defn}

\begin{defn}[Transition Map]
  Let $M$ be a topological $n$-manifold and $(U, \phi), (V, \psi)$ be coordinate charts such that $U \cap V \ne \emptyset$.
  $\psi \circ \phi^{-1}: \phi(U \cap V) \mapsto \psi(U \cap V)$ is called a \textit{transition map} from $\phi$ to $\psi$.
\end{defn}

\subsection{Smooth Manifolds}

\begin{defn}[Smoothly Compatible]
  Let $M$ be a topological $n$-manifold.
  Two coordinate charts $(U, \phi), (V, \psi)$ are called \textit{smoothly compatible} if $U \cap V = \emptyset$ or the transition map $\psi \circ \phi^{-1}$ is a diffeomorphism.
\end{defn}

\begin{defn}[Smooth Atlas]
  Let $M$ be a topological $n$-manifold.
  A \textit{smooth atlas} is an atlas $\mathcal{A}$ such that any two charts in $\mathcal{A}$ are smoothly compatible with each other.
\end{defn}

\begin{defn}[Smooth Structure]
  If $M$ is a topological $n$-manifold, an atlas $\mathcal{A}$ that is not properly contained in any larger smooth atlas is called \textit{maximal} or a \textit{smooth structure on $M$}
\end{defn}

\begin{defn}[Smooth Manifold]
  A \textit{smooth manifold} is a topological manifold equipped with a smooth structure.
\end{defn}

\begin{defn}
  Suppose $(M, \mathcal{A})$ is a smooth manifold.
  \begin{itemize}
    \item
      Any chart $(U, \phi) \in \mathcal{A}$ is called a \textit{smooth chart}.
    \item
      Given a smooth chart $(U, \phi)$, $U$ is called a \text{smooth coordinate domain} and $\phi$ is called a \textit{smooth coordinate map}.
    \item
      Given a smooth chart $(U, \phi)$, $U$ is called a \textit{smooth coordinate ball} if it is a coordinate ball.
  \end{itemize}
\end{defn}

\begin{rem}
  One must define a smooth structure on a topological manifold before talking about a smooth chart.
\end{rem}
