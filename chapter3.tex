\subsection{Exercises}
\begin{customexer}{3.5(Proof of Lemma 3.4)}
  Suppose $M$ is a smooth manifold with or without boundary, $p \in M, v \in T_pM$, and $f, g \in C^{\infty}(M)$.
  \begin{enumerate}[label=(\alph*)]
    \item 
      If $f$ is a constant function, then $vf = 0$.
    \item
      If $f(p) = g(p) = 0$, then $v(fg) = 0$.
  \end{enumerate}
\end{customexer}

\begin{proof}
  $ $
  \begin{enumerate}[label=(\alph*)]
    \item
      Let $h$ be the constant function that always takes the value 1.
      Then $f(p) = ch(p)$ for some $c \in \mathbb{R}$.
      Then $v(ff) = f(p)vf + f(p)vf$, so $c^2v(h) = c^2v(h) + c^2v(h)$.
      Therefore, $c^2v(h) = 0$, so $cv(h) = 0$.
      Since $v$ is linear, this implies $0 = v(ch) = v(f)$, so $v(f) = 0$.
    \item
      $v(fg) = f(p)vg + g(p)vf = 0 + 0 = 0$.
  \end{enumerate}
\end{proof}

\begin{customexer}{3.7(Proof of Proposition 3.6)}
  Let $M, N$, and $P$ be smooth manifolds with or without boundary, let $F: M \rightarrow N$ and $G: N \rightarrow P$ be smooth maps, and let $p \in M$.
  \begin{enumerate}[label=(\alph*)]
    \item
      $dF_p:T_pM \rightarrow T_{F(p)}N$ is linear.
    \item
      $d(G \circ F)_p = dG_{F(p)} \circ dF_p: T_pM \rightarrow T_{G \circ F(p)} P$.
    \item
      $d(\Id_M)_p = \Id_{T_pM}: T_pM \rightarrow T_pM$.
    \item
      If $F$ is a diffeomorphism, then $dF_p: T_pM \rightarrow T_{F(p)}N$ is an isomorphism, and $(dF_p)^{-1} = d(F^{-1})_{F(p)}$.
  \end{enumerate}
\end{customexer}

\begin{proof}
  \begin{enumerate}[label=(\alph*)]
    \item
      $\forall v, w \in T_pM, \forall c \in \mathbb{R}, \forall f \in C^{\infty}(N)$,
      \begin{align*}
        dF_p(cv + w)(f)
          &= (cv + w)(f \circ F) \\
          &= (cv)(f \circ F) + w(f \circ F) \\
          &= c(v(f \circ F)) + w(f \circ F) \\
          &= c(dF_p(v)(f)) + dF_p(w)(f) \\
          &= (cdF_p(v))(f) + dF_p(w)(f) \\
          &= (cdF_p(v) + dF_p(w))(f).
      \end{align*}
      Therefore, $dF_p(cv + w) = cdF_p(v) + dF_p(w)$.
    \item
      $\forall v \in T_pM, f \in C^{\infty}(P)$,
      \begin{align*}
        d(G \circ F)_p(v)(f)
          &= v(f \circ (G \circ F)) \\
          &= v((f \circ G) \circ F) \\
          &= (dF_p(v))(f \circ G) \\
          &= (dG_{F(p)}(dF_p(v)))(f) \\
          &= ((dG_{F(p)} \circ dF_p)(v))(f) \\
      \end{align*}
      Therefore, $d(G \circ F)_p = dG_{F(p)} \circ dF_p$.
    \item
      $\forall v \in T_p(M), \forall f \in C^{\infty}(M)$,
      \begin{align*}
        d(\Id_M)_p(v)(f)
          &= v(f \circ \Id_M) \\
          &= v(f).
      \end{align*}
      Therefore, $d(\Id_M)_p(v) = v$, so $d(\Id_M)_p = \Id_{T_pM}$.
    \item
      $F^{-1}$ exists and it is a smooth map since $F$ is a diffeomorphism.
      By combining (b) and (c), we obtain $dF_p$ and $dF^{-1}_{F(p)}$ are the inverse of each other.
      Therefore, $dF_p$ is an isomorphism.
  \end{enumerate}
\end{proof}

\subsection{Problems}

\begin{customprob}{3.2(Proof of Proposition 3.14)}\label{problem_3_2}
  Let $M_1, \cdots, M_k$ be smooth manifolds, and for each $j$, let $\pi_j: M_1 \times \cdots \times M_k \rightarrow M_j$ be the projection onto the $M_j$ factor.
  For any point $p = (p_1, \cdots, p_k) \in M_1 \times \cdots \times M_k$, the map
  \begin{align*}
    \alpha: T_p(M_1 \times \cdots \times M_k) \rightarrow T_{p_1}M_1 \oplus \cdots \times T_{p_k}M_k
  \end{align*}

  defined by

  \begin{align*}
    \alpha(v) = (d(\pi_1)_p(v), \cdots, d(\pi_k)_p(v))
  \end{align*}

  is an isomorphism.
  The same is true if one of the spaces $M_i$ is a smooth manifold with boundary.
\end{customprob}

\begin{proof}
  \todo[inline,caption={}]{
    Solve this!
  }
\end{proof}
