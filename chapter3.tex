\begin{customexer}{3.5(Proof of Lemma 3.4)}
  Suppose $M$ is a smooth manifold with or without boundary, $p \in M, v \in T_pM$, and $f, g \in C^{\infty}(M)$.
  \begin{enumerate}[label=(\alph*)]
    \item 
      If $f$ is a constant function, then $vf = 0$.
    \item
      If $f(p) = g(p) = 0$, then $v(fg) = 0$.
  \end{enumerate}
\end{customexer}

\begin{proof}
  $ $
  \begin{enumerate}[label=(\alph*)]
    \item
      Let $h$ be the constant function that always takes the value 1.
      Then $f(p) = ch(p)$ for some $c \in \mathbb{R}$.
      Then $v(ff) = f(p)vf + f(p)vf$, so $c^2v(h) = c^2v(h) + c^2v(h)$.
      Therefore, $c^2v(h) = 0$, so $cv(h) = 0$.
      Since $v$ is linear, this implies $0 = v(ch) = v(f)$, so $v(f) = 0$.
    \item
      $v(fg) = f(p)vg + g(p)vf = 0 + 0 = 0$.
  \end{enumerate}
\end{proof}
