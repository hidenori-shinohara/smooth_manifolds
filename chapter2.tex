\subsection{Exercises}
\begin{customexer}{2.1}
  Let $M$ be a smooth manifold with or without boundary.
  Show that pointwise multiplication turns $C^{\infty}(M)$ into a commutative ring and a commutative and associative algebra over $\mathbb{R}$.
\end{customexer}

\begin{proof}
  $ $
  \begin{itemize}
    \item
      The constant map $f(p) = 0$ is clearly in $C^{\infty}(M)$ and it is the additive identity.
    \item
      The constant map $f(p) = 1$ is clearly in $C^{\infty}(M)$ and it is the multiplicative identity.
    \item
      Let $f \in C^{\infty}(M), g \in C^{\infty}(M)$.
      Let $p \in M$ and $(\phi, U)$ be a smooth chart for $p$.
      Then $f \circ \phi^{-1}$ and $g \circ \phi^{-1}$ are both smooth(Exercise \ref{ex_2_3}), real-valued maps defined on an open subset of $\mathbb{R}^n$.
      Thus $f + g$ is in $C^{\infty}(M)$ 
      Moreover, $f + g = g + f$ because addition in $\mathbb{R}$ is commutative.
    \item
      Let $f, g, h \in C^{\infty}(M)$.
      Let $p \in M$ and $(\phi, U)$ be a smooth chart for $p$.
      Then $f \circ \phi^{-1}$ and $g \circ \phi^{-1}$ are both smooth(Exercise \ref{ex_2_3}), real-valued maps defined on an open subset of $\mathbb{R}^n$.
      Therefore, $fg$ is in $C^{\infty}(M)$
      Moreover, $fg = gf$ and $(fg)h = f(gh)$ because multiplication in $\mathbb{R}$ is commutative and associative.
    \item
      Let $c \in \mathbb{R}, f \in C^{\infty}(M)$.
      Then $cf$ can be seen as $fg$ where $g$ is the constant function whose value is $c$.
      As shown above, $cf \in C^{\infty}(M)$.
  \end{itemize}
\end{proof}

\begin{customexer}{2.2}
  Let $U$ be an open submanifold of $\mathbb{R}^n$ with its standard smooth manifold structure.
  Show that a function $f: U \rightarrow \mathbb{R}^k$ is smooth in the sense just defined if and only if it is smooth in the sense of ordinary calculus.
  Do the same for an open submanifold with boundary in $\mathbb{H}^n$.
\end{customexer}

\begin{proof}
  $f$ is smooth in the sense just defined if and only if $f \circ \Id^{-1}$ is smooth in the sense of ordinary calculus.
  Since $f \circ \Id^{-1} = f$, $f \circ \Id^{-1}$ is smooth in the sense of ordinary calculus if and only if $f$ is smooth in the sense of ordinary calculus.
\end{proof}

\begin{customexer}{2.3}\label{ex_2_3}
  Let $M$ be a smooth manifold with or without boundary, and suppose $f: M \rightarrow \mathbb{R}^k$ is a smooth function.
  Show that $f \circ \phi^{-1}: \phi(U) \rightarrow \mathbb{R}^k$ is smooth for every smooth chart $(U, \phi)$ for $M$.
\end{customexer}

\begin{proof}
  Let $\phi(x) \in \phi(U)$.
  Since $f$ is smooth, there exists $(V, \psi)$ such that $f \circ \psi^{-1}: \psi(V) \rightarrow \mathbb{R}^k$ is smooth and $x \in V$.
  Let $W = U \cap V$.
  Then $f \circ \psi^{-1}: \psi(W) \rightarrow \mathbb{R}^k$ is smooth and $\psi \circ \phi^{-1}: \phi(W) \rightarrow \psi(W)$ is a diffeomorphism where $\phi(W)$ is a neighborhood of $W$.
  Then the restriction of $f \circ \psi^{-1}$ to $\phi(W)$ is identical to $(f \circ \psi^{-1}) \circ (\psi \circ \phi^{-1})$.
  Since he composition of a smooth function is smooth, $f \circ \psi^{-1}$ is smooth.
\end{proof}

\begin{customexer}{2.7(Prove Proposition 2.5)}\label{proof_prop_2_5}
  Suppose $M$ and $N$ are smooth manifolds with or without boundary, and $F: M \rightarrow N$ is a map.
  Then $F$ is smooth if and only if either of the following conditions is satisfied:
  \begin{enumerate}[label=(\alph*)]
    \item 
      For every $p \in M$, there exist smooth charts $(U, \phi)$ containing $p$ and $(V, \psi)$ containing $F(p)$ such that $U \cap F^{-1}(V)$ is open in $M$ and the composite map $\psi \circ F \circ \phi^{-1}$ is smooth from $\phi(U \cap F^{-1}(V))$ to $\psi(V)$.
    \item
      $F$ is continuous and there exist smooth atlases $\{ (U_{\alpha}, \phi_{\alpha}) \}$ and $\{ (V_{\beta}, \psi_{\beta}) \}$ for $M$ and $N$, respectively, such that for each $\alpha$ and $\beta$, $\psi_{\beta} \circ F \circ \phi_{\alpha}^{-1}$ is a smooth map from $\phi_{\alpha}(U_{\alpha} \cap F^{-1}(V_{\beta}))$ to $\psi_{\beta}(V_{\beta})$.
  \end{enumerate}
\end{customexer}

\begin{proof}
  Let $\mathcal{A}_M$ and $\mathcal{A}_N$ be smooth structures of $M$ and $N$.
  Suppose $F$ is smooth.
  By Proposition 2.4, $F$ is continuous.
  For every $p \in M$ there exist coordinate charts $(U_p, \phi_p)$ containing $p$ and $(V_p, \psi_p)$ containing $F(p)$ such that $F(U_p) \subset V_p$ and $\psi_p \circ F_p \circ \phi_p^{-1}$ is smooth from $\phi_p(U_p)$ to $\psi_p(V_p)$.
  Then $\{ (U_p, \phi_p) \mid p \in M \} \subset \mathcal{A}_M$ and $A_n \{ (V_p, \psi_p) \mid p \in M \} \subset \mathcal{A}_N$ are smooth atlases.
  Moreover, for every $(U_p, \phi_p)$ and $(V_q, \psi_q)$, $\psi_q \circ F \circ \phi_p^{-1}$ is a smooth map from $\phi_p(U_p \cap F^{-1}(V_q))$ to $\psi_q(V_q)$ because $\psi_q \circ F \circ \phi^{-1}_p = (\psi_q \circ \psi_p^{-1}) \circ (\psi_p \circ F \circ \phi_p^{-1})$ where $\psi_q \circ \psi_q^{-1}$ and $\psi_p \circ F \circ \phi_p^{-1}$ are smooth.
  Therefore, the definition implies (b).

  (b) implies (a) because if $F$ is continuous, $F^{-1}(V_{\beta})$ is open in $M$ for every $\beta$, so $U \cap F^{-1}(V)$ is open in $M$.

  Finally, we show that (a) implies the definition.
  Suppose $F$ satisfies (a).
  Let $p \in M$.
  Let $(U, \phi) \in \mathcal{A}_M$ and $(V, \psi) \in \mathcal{A}_N$ be smooth charts satisfying the properties described in (a).
  Let $U' = U \cap F^{-1}(V)$ and consider $(U', \phi\mid_{U'})$.
  Then $(U', \phi\mid_{U'}) \in \mathcal{A}_M$ because it must be smoothly compatible with any other smooth coordinate chart in $\mathcal{A}_M$.
  Moreover, $F(U') \subset V$ and $\psi \circ F \circ (\phi\mid_{U'})^{-1}:\phi(U') \rightarrow \psi(V)$ is smooth.
  Therefore, (a) implies the definition.

  Hence, (a), (b) and the definition are all equivalent.
\end{proof}

\begin{customexer}{2.7(Proof of Proposition 2.6)}\label{proof_prop_2_6}
  Let $M$ and $N$ be smooth manifolds with or without boundary, and let $F: M \rightarrow N$ be a map.
  \begin{enumerate}[label=(\alph*)]
    \item 
      If every point $p \in M$ has a neighborhood $U$ such that the restriction $F\vert_U$ is smooth, then $F$ is smooth.
    \item
      Conversely, if $F$ is smooth, then its restriction to every open subset is smooth.
  \end{enumerate}
\end{customexer}

\begin{proof}
  Let $\mathcal{A}_M, \mathcal{A}_N$ be smooth structures of $M, N$, respectively.
  \begin{enumerate}[label=(\alph*)]
    \item 
      Let $p \in M$.
      Let $U$ be a neighborhood of $p$ such that $F\vert_U$ is smooth.
      By \ref{exercise_1_44}, $U$ is a smooth manifold with the induced smooth structure $\mathcal{A}_U = \{ (V, \phi) \in \mathcal{A}_M \mid V \subset U \}$.
      Since $F\vert_U$ is smooth, there exist $(V, \phi) \in \mathcal{A}_U$ and $(W, \psi) \in \mathcal{A}_N$ such that:
      \begin{itemize}
        \item
          $F\vert_U(V) \subset W$.
        \item
          $\psi \circ F\vert_U \circ \phi^{-1}:\phi(V) \rightarrow \psi(W)$ is smooth.
      \end{itemize}
      Since $V \subset U$, $F(V) \subset W, \psi \circ F \circ \phi^{-1}:\phi(V) \rightarrow \psi(W)$ is smooth, and $(V, \phi) \in \mathcal{A}$.
      Therefore, $F$ is smooth.
    \item
      Let $U \subset M$ be an open subset.
      By \ref{exercise_1_44}, $U$ is a smooth manifold with the induced smooth structure $\mathcal{A}_U = \{ (V, \phi) \in \mathcal{A}_M \mid V \subset U \}$.
      Let $p \in U$.
      Then $p \in F$, so there exist $(V, \phi) \in \mathcal{A}_M, (W, \psi) \in \mathcal{A}_N$ such that $F(V) \subset W$ and $\psi \circ F \circ \phi^{-1}: \phi(V) \rightarrow \psi(W)$ is smooth.
      Then $(V \cap U, \phi\vert_{V \cap U})$ is a chart that is smoothly compatible with every chart in $\mathcal{A}_M$.
      Therefore, $(V \cap U, \phi\vert_{V \cap U}) \in \mathcal{A}_M$.
      Moreover, $\phi\vert_{V \cap U}(V \cap U) \subset \phi(V) \subset W$ and $\psi \circ F \circ (\phi\vert_{V \cap U}(V \cap ))^{-1}$ is clearly smooth.
      Therefore, $F \vert_U$ is smooth.
  \end{enumerate}
\end{proof}

\begin{customexer}{2.9}
  Suppose $F: M \rightarrow N$ is a smooth map between smooth manifolds with or without boundary.
  Show that the coordinate representation of $F$ with respect to \textit{every} pair of smooth charts for $M$ and $N$ is smooth.
\end{customexer}

\begin{proof}
  Let $(M, \mathcal{A}_M), (N, \mathcal{A}_N)$ be smooth manifolds with or without boundary.
  Let $F: M \rightarrow N$ be a smooth map.
  Let $(U, \phi) \in \mathcal{A}_M, (V, \psi) \in \mathcal{A}_N$ be given.
  We must show that $\hat{F} = \psi \circ F \circ \phi^{-1}$ is a smooth function from $\phi(U \cap F^{-1}(V))$ to $\psi(V)$.
  Let $\phi(p) \in \phi(U \cap F^{-1}(V))$.
  Then $p \in M$, so there exist $(U_0, \phi_0) \in \mathcal{A}_M$ and $(V_0, \psi_0) \in \mathcal{A}_N$ such that
  \begin{itemize}
    \item
      $p \in U_0 \subset U \cap F^{-1}(V)$;
    \item
      $\phi_0(U_0) \subset V_0$;
    \item
      $\psi_0 \circ F \circ \phi_0^{-1}:\phi_0(U_0) \rightarrow \psi(V_0)$ is smooth.
  \end{itemize}
  Then $\psi \circ F \circ \phi^{-1} \vert_{\phi(U_0)} = (\psi \circ \psi_0^{-1}) \circ (\psi_0 \circ F \circ \phi_0^{-1}) \circ (\phi_0 \circ \phi)$.
  Since the composition of smooth functions in Euclidean spaces is smooth, $\hat{F}$ is smooth.
\end{proof}

\begin{customexer}{2.11(Proof of Proposition 2.10)}\label{exercise_2_11}
  Let $M, N$ and $P$ be smooth manifolds with or without boundary.
  \begin{enumerate}[label=(\alph*)]
    \item 
      Every constant map $c: M \rightarrow N$ is smooth.
    \item
      The identity map of $M$ is smooth.
    \item
      If $U \subset M$ is an open submanifold with or without boundary, then the inclusion map $U \rightarrow M$ is smooth.
  \end{enumerate}
\end{customexer}

\begin{proof}
  Let $\mathcal{A}_M, \mathcal{A}_N, \mathcal{A}_P$ be smooth structures of $M, N, P$, respectively.
  \begin{enumerate}[label=(\alph*)]
    \item 
      $F$ is clearly continuous.
      Moreover, for every $(U_{\alpha}, \phi_{\alpha}) \in \mathcal{A}_M, (V_{\beta}, \psi_{\beta}) \in \mathcal{A}_N$, $\psi_{\beta} \circ F \circ \phi_{\alpha}^{-1}$ is a constant map, so it is smooth.
      By (\ref{proof_prop_2_5}), $F$ is smooth.
    \item
      Let $p \in M$.
      Choose $(U, \phi) \in \mathcal{A}_M$ such that $p \in U$.
      Then $F(U) \subset U$ and $\phi \circ F \circ \phi^{-1} = \Id_U$, so it is smooth.
      Therefore, $F$ is smooth.
    \item
      By \ref{exercise_1_44}, $\mathcal{A}_U = \{ (V, \phi) \mid V \subset U \}$ is a smooth structure of $U$.
      Let $p \in U$.
      Then $p \in V$ for some $(V, \phi) \in \mathcal{A}_U$.
      Then $(V, \phi) \in \mathcal{A}_M$, trivially.
      Since $F(V) \subset V$ and $\phi \circ F \circ \phi^{-1}$ is simply the identity map on $V$, $F$ is smooth.
  \end{enumerate}
\end{proof}

\begin{customexer}{2.16(Proof of Proposition 2.15)}\label{exercise_2_16}
  $ $
  \begin{enumerate}[label=(\alph*)]
    \item 
      Every composition of diffeomorphisms is a diffeomorphism.
    \item
      Every finite product of diffeomorphisms between smooth manifolds is a diffeomorphism.
    \item
      Every diffeomorphism is a homeomorphism and an open map.
    \item
      The restriction of a diffeomorphism to an open submanifold with or without boundary is a diffeomorphism onto its image.
    \item
      ``Diffeomorphic" is an equivalence relation on the class of all smooth manifolds with or without boundary.
  \end{enumerate}
\end{customexer}

\begin{customexer}{2.16(Proof of Proposition 2.15)}
  Let $(M, \mathcal{A}_M), (N, \mathcal{A}_N), (P, \mathcal{A}_P)$ be smooth manifolds with or without boundary, and let $F: M \rightarrow N, G: N \rightarrow P$ be diffeomorphisms.
  \begin{enumerate}[label=(\alph*)]
    \item 
      By Proposition 2.10(d), $G \circ F$ and $F^{-1} \circ G^{-1}$ are smooth.
      Then $(G \circ F) \circ (F^{-1} \circ G^{-1})$ and $(F^{-1} \circ G^{-1}) \circ (G \circ F)$ are both the identity map on the corresponding space, so $F^{-1} \circ G^{-1}$ is the smooth inverse of $G \circ F$.
      Therefore, $G \circ F$ is a diffeomorphism.
    \item
      By Example 1.34, we know that $M_1 \times \cdots \times M_k$ and $N_1 \times \cdots \times N_k$ are both smooth manifolds.
      Let $\mathcal{A}_{M_i}, \mathcal{A}_{N_i}, \mathcal{A}_M$ and $\mathcal{A}_N$ denote the smooth manifold structures of $M_i, N_i, M_1 \times \cdots \times M_k, N_1 \times \cdots \times N_k$, respectively.
      Let a smooth map $F_i: M_i \rightarrow N_i$ be given for each $i$.
      Let $(p_1, \cdots, p_k) \in M_1 \times \cdots M_k$ be given.
      Then there exist $(U_i, \phi_i) \in \mathcal{A}_{M_i}$ and $(V_i, \psi_i) \in \mathcal{A}_{N_i}$ such that $p_i \in U_i, F_i(U_i) \subset V_i, \psi_i \circ F_i \circ \phi_i^{-1}: \phi_i(U_i) \rightarrow \psi_i(V_i)$ is smooth for each $i$.
      This implies that $(\psi_1 \circ F_1 \circ \phi_1^{-1}) \times \cdots (\psi_k \circ F_k \circ \phi_k^{-1}) = (\psi_1 \times \cdots \times \psi_k) \circ (F_1 \times \cdots \times F_k) \circ (\phi_1 \times \cdots \times \phi_k)^{-1}$ is smooth.

      Therefore, $F_1 \times \cdots \times F_k$ is smooth.
      Using the exact same argument, we can conclude that $F_1^{-1} \times \cdots \times F_k^{-1}$ is smooth.
      Since $(F_1 \times \cdots \times F_k)^{-1} = F_1^{-1} \times \cdots \times F_k^{-1}$, $F_1 \times \cdots \times F_k$ is a diffeomorphism.
    \item
      Proposition 2.4 states that every smooth map is continuous.
      Thus $F$ and $F^{-1}$ are both continuous.
      Therefore, $F$ is a homeomorphism and also an open map.
    \item
      Let $U \subset M$ be an open subset.
      By (\ref{proof_prop_2_6}), $F\vert_U$ is smooth.
      Since $F$ is a homeomorphism as shown in (c), $F(U)$ is an open subset of $N$.
      Therefore, $F^{-1}\vert_{F(U)}$ is smooth by (\ref{proof_prop_2_6}).
      Clearly, $F\vert_U$ and $F^{-1}\vert_{F(U)}$ are the inverse of each other.
      Therefore, $F\vert_U$ is a diffeomorphism.
    \item
      By (\ref{exercise_2_11}), the identity map on $M$ is a diffeomorphism, so the reflexive property is satisfied.
      Moreover, $(F^{-1})^{-1} = F$, so the symmetric property is satisfied.
      By (a), the composition of two diffeomorphisms is a diffeomorphism, so the transitive property is satisfied.
      Therefore, ``diffeomorphic" is an equivalence relation.
  \end{enumerate}
\end{customexer}

\begin{customexer}{2.19(Proof of Theorem 2.18)}
  Suppose $M$ and $N$ are smooth manifolds with boundary and $F: M \rightarrow N$ is a diffeomorphism.
  Then $F(\partial M) = \partial N$, and $F$ restricts to a diffeomorphism from $\Int M$ to $\Int N$.
\end{customexer}

\begin{proof}
  Let $\mathcal{A}_M, \mathcal{A}_N$ denote the smooth structures of $M, N$, respectively.
  Let $p \in \partial M$.
  Then there exists a chart containing $p$ that sends $p$ to $\partial \mathbb{H}^n$.
  By Theorem 1.46, every chart containing $p$ sends $p$ to $\partial \mathbb{H}^n$.

  Since $F$ is smooth, there exist $(U, \phi) \in \mathcal{A}_M, (V, \psi) \in \mathcal{A}_N$ such that $F(U) \subset V$ and $\psi \circ F \circ \phi^{-1}$ is a smooth map from $\phi(U)$ to $\psi(V)$.
  $F^{-1}$ is a homeomorphism by (\ref{exercise_2_16}).
  Then $(\phi^{-1} \circ F^{-1}, F(U))$ is a coordinate chart around $F(p)$ because we obtain a homeomorphism by restricting the composition of two injective continuous maps to its image.
  Moreover, we claim that $(\phi^{-1} \circ F^{-1}, F(U))$ is smoothly compatible with every chart in $\mathcal{A}_N$.
  Let $(\psi_1, V_1) \in \mathcal{A}_N$ be given.
  Then $(\phi^{-1} \circ F^{-1}) \circ \psi_1^{-1} = (\phi^{-1} \circ F^{-1} \circ \psi^{-1}) \circ (\psi \circ \psi_1^{-1})$, and the composition of two smooth maps is smooth.
  Therefore, $(\phi^{-1} \circ F^{-1}, F(U)) \in \mathcal{A}_N$, and this chart contains $F(p)$ and sends $F(p)$ to $\partial \mathbb{H}^n$.
  In other words, $F(p) \in \partial N$.

  Since $F^{-1}$ is also smooth, $F^{-1}(\partial N) \subset \partial M$.
  $F^{-1}(\partial N) \subset \partial M \implies F(F^{-1}(\partial N)) \subset F(\partial M) \subset \partial N$.
  Since $F$ is a bijection, $F(F^{-1}(\partial N)) = \partial N$.
  Therefore, $F(\partial M) = \partial N$.

  This implies that $F(\Int M) = \Int N$.
  By (\ref{exercise_1_44}(c)) and (\ref{exercise_2_16}(d)), $F$ is a diffeomorphism between $\Int M$ and $\Int N$.
\end{proof}

\subsection{Problems}

\begin{customprob}{2-1}
  Define $f: \mathbb{R} \rightarrow \mathbb{R}$ by
  \begin{align*}
    f(x) &= \begin{cases}
      1, & x \geq 0, \\
      0, & x < 0.
    \end{cases}
  \end{align*}
  Show that for every $x \in \mathbb{R}$, there are smooth coordinate charts $(U, \phi)$ containing $x$ and $(V, \psi)$ containing $f(x)$ such that $\psi \circ f \circ \phi^{-1}$ is smooth as a map from $\phi(U \cap f^{-1}(V))$ to $\psi(V)$, but $f$ is not smooth in the sense we defined in this chapter.
\end{customprob}

\begin{proof}
  $\phi = \psi = \Id$ in this solution.

  If $x \geq 0$, then let $U = \mathbb{R}, V = (0, \infty)$.
  Then $\phi(U \cap f^{-1}(V)) = [0, \infty)$.
  Thus $\psi \circ f \circ \phi^{-1}: [0, \infty) \rightarrow (0, \infty)$ is the constant map that sends every number to 1.
  Therefore, it is smooth.

  If $x < 0$, then let $U = \mathbb{R}, V = (-\infty, 1)$.
  Then $\phi(U \cap f^{-1}(V)) = (-\infty, 0)$.
  Thus $\psi \circ f \circ \phi^{-1}: (-\infty, 0) \rightarrow (-\infty, 1)$ is the constant map that sends every number to 0.
  Therefore, it is smooth.

  It might seem that we can apply (\ref{proof_prop_2_5}) to show that $f$ is smooth, but (\ref{proof_prop_2_5}) requires that $U \cap f^{-1}(V)$ be open in $M$.

  $f$ maps the interval $(-1, 1)$ to $\{ 0, 1 \}$.
  Since the image of a connected set under a continuous map must be connected, $f$ cannot be continuous.
  By Proposition 2.4, $f$ cannot be smooth.
\end{proof}

\begin{customprob}{2-2(Proof of Proposition 2.12)}\label{problem_2_2}
  Suppose $M_1, \cdots, M_k$ and $N$ are smooth manifolds with or without boundary, such that at most one of $M_1, \cdots, M_k$ has nonempty boundary.
  For each $i$, let $\pi_i: M_1 \times \cdots \times M_k \rightarrow M_i$ denote the projection onto the $M_i$ factor.
  A map $F: N \rightarrow M_1 \times \cdots \times M_k$ is smooth if and only if each of the component maps $F_i = \pi_i \circ F: N \rightarrow M_i$ is smooth.
\end{customprob}

\begin{proof}
  Let $\mathcal{A}_{M_1}, \cdots, \mathcal{A}_{M_k}, \mathcal{A}_N$ be the smooth structures of $M_1, \cdots, M_k, N$.
  Let $d_1, \cdots, d_k$ denote the dimensions of $M_1, \cdots, M_n$, respectively.
  Let $d = \sum d_i$.

  First, suppose that $F$ is smooth.
  By (\ref{exercise_2_11}), the composition of smooth maps is smooth.
  Thus it suffices to show that $\pi_i: M_1 \times \cdots \times M_k \rightarrow M_i$ is smooth for each $i$.
  We show that $\pi_1$ is smooth and the other cases can be shown similarly.

  Let $(x_1, \cdots, x_k) \in M_1 \times \cdots \times M_k$.
  Then for each $i$, there exist $(U_i, \phi_i) \in \mathcal{A}_{M_i}$ and $(V_i, \psi_i) \in \mathcal{A}_{M_i}$ such that $x_i \in U_i$ and $\phi_i(U_i) \subset V_i$.
  Then we have $(\phi_1 \times \cdots \times \phi_k)(U_1 \times \cdots \times U_k) \subset V_1 \times \cdots \times V_k$ and the composition $\phi_i \circ \pi_1 \circ (\phi_1 \times \cdots \times \phi_k)^{-1}$ is the projection of the first $d_1$ coordinates from $\mathbb{R}^n$ onto $\mathbb{R}^{d_1}$.
  Therefore, it is clearly smooth, so $\pi_1$ is smooth.

  Suppose each $F_i = \pi_i \circ F: N \rightarrow M_i$ is smooth.
  Let $p \in N$.
  Then for each $i$, there exist $(U_i, \phi_i) \in \mathcal{A}_N$ and $(V_i, \psi_i) \in \mathcal{A}_{M_i}$ such that $p \in U_i, F_i(U_i) \subset V_i$ and $\psi_k \circ F_i \circ \phi_i^{-1}$.
  Let $U = U_1 \cap \cdots \cap U_k$.
  $U$ is a neighborhood of $p$ and the restriction of $\phi_1$ to $U$ is a homeomorphism.
  Then we claim that $(\phi_1, U) \in \mathcal{A}_N$ and $(\psi_1 \times \cdots \times \psi_k, V_1 \times \cdots \times V_k) \in \mathcal{A}_{M_1 \times \cdots \times M_k}$ are charts that satisfy the necessary properties.
  \begin{itemize}
    \item
      $F(U) \subset V_1 \times \cdots \times V_k$.
    \item
      For each $i$, $\psi_i \circ F_i \circ \phi_1^{-1} = (\psi_i \circ F_i \circ \phi_i^{-1}) \circ (\phi_i \circ \phi_1^{-1}): \phi_1(U) \rightarrow \psi_i(V_i)$ is smooth because the composition of two smooth maps is smooth.
      Thus $(\psi_1 \circ F_1 \circ \phi_1^{-1}) \times \cdots \times (\psi_k \circ F_k \circ \phi_1^{-1}): \phi_1(U) \rightarrow \psi_1(V_1) \times \cdots \times \psi_k(V_k)$ is smooth.
      Moreover, $(\psi_1 \times \cdots \times \psi_k) \circ F \circ \phi_1^{-1} = (\psi_1 \circ F_1 \circ \phi_1^{-1}) \times \cdots \times (\psi_k \circ F_k \circ \phi_1^{-1})$.
  \end{itemize}

  Therefore, $F$ is smooth.
\end{proof}

\begin{customprob}{2-3}\label{problem_2_3}
  For each of the following maps between spheres, compute sufficiently many coordinate representations to prove that it is smooth.

  \begin{enumerate}[label=(\alph*)]
    \item 
      $p_n: S^1 \rightarrow S^1$ is the $n$th power map for $n \in \mathbb{Z}$, given in complex notation by $p_n(z) = z^n$.
    \item
      $\alpha: S^n \rightarrow S^n$ is the antipodal map $\alpha(x) = -x$.
    \item
      $F: S^3 \rightarrow S^2$ is given by $F(w, z) = (z\overline{w} + w\overline{z}, iw\overline{z} - iz\overline{w}, z\overline{z} - w\overline{w})$ where we think of $S^3$ as the subset $\{ (w, z) : \abs{w}^2 + \abs{z}^2 = 1 \}$ of $\mathbb{C}^2$.
  \end{enumerate}
\end{customprob}

\begin{proof}
  $ $
  \begin{enumerate}[label=(\alph*)]
    \item 
      Example 1.31 shows the existence of a smooth structure of $S^1$ and let $\mathcal{A}$ denote it.
      Let $p \in S^1$.
      Then there exist $(U_i^{\pm}, \phi_i^{\pm}), (U_j^{\pm}, \phi_j^{\pm}) \in \mathcal{A}$ around $p, p_n(p)$, respectively.
      Then the composition $\phi_j^{\pm} \circ f \circ (\phi_i^{\pm})^{-1}$ is equal to one of $\cos(n(\arccos(x))), \sin(n(\arcsin(x))), \cos(n(\arcsin(x))), \sin(n(\arccos(x)))$, all of which are clearly smooth.
      By Proposition 2.5(a), $p_n$ is smooth.
    \item
      Example 1.31 shows the existence of a smooth structure of $S^n$ and let $\mathcal{A}$ denote it.
      Let $p \in S^1$.
      Then there exists a chart $(U_i^{\pm}, \phi_i^{\pm})$ in $\mathcal{A}$ around $p$.
      Then $(U_i^{\mp}, \phi_i^{\mp})$ is a chart containing $\alpha(p)$ with $\alpha(U_i^{\pm}) \subset U_i^{\mp}$.
      Then $\phi_i^{\mp} \circ \alpha \circ \phi_i^{\pm}$ is the map $x \mapsto -x$, which is clearly smooth.
    \item
      Let $z = a + bi, w = c + di$.
      $z\overline{w} = ac + bd + i(bc - ad)$ and $w\overline{z} = (ac + bd) - i(bc - ad)$.
      Then $z\overline{w} + w\overline{z} = 2(ac + bd) = 2\Re(z\overline{w})$ and $i(w\overline{z} - z\overline{w}) = 2\Im(z\overline{w})$.
      \begin{align*}
        (2\Re(z\overline{w}))^2 + (2\Im(z\overline{w}))^2 + (\abs{z}^2 - \abs{w}^2)^2
          &= 4\abs{z\overline{w}}^2 + (\abs{z}^2 - \abs{w}^2)^2 \\
          &= 4\abs{z}^2\abs{\overline{w}}^2 + (\abs{z}^2 - \abs{w}^2)^2 \\
          &= (\abs{z}^2 + \abs{w}^2)^2 \\
          &= 1.
      \end{align*}
      Therefore, $F$ indeed maps $S^3$ into $S^2$.
      Moreover, this map is continuous.
      Let $(z = a + bi, w = c + di) \in S^3$ be given.
      Suppose that $(U_4^{+}, \phi_4^{+})$ and $(V_3^{+}, \psi_3^{+})$ are charts containing $(z, w)$ and $F(z, w)$.
      Then $\psi_3^{+} \circ F \circ \phi_4^{+}: (a, b, c) \mapsto (2u, 2v)$ where $u + iv = (a + bi)(c - \sqrt{1 - a^2 - b^2 - c^2}i)$ which is a smooth map from $\phi_4^{+}(U_4^{+}) \subset \mathbb{R}^3$ into $\mathbb{R}^2$.
      Other cases are similar, and thus $F$ is smooth by Proposition 2.5(b).
  \end{enumerate}
\end{proof}

\begin{customprob}{2-5}\label{problem_2_5}
  Let $\mathbb{R}$ be the real line with its standard smooth structure, and let $\tilde{R}$ denote the same topological manifold with the smooth structure defined in Example 1.23.
  Let $f: \mathbb{R} \rightarrow \mathbb{R}$ be a function that is smooth in the usual sense.
  \begin{enumerate}[label=(\alph*)]
    \item 
      Show that $f$ is also smooth as a map from $\mathbb{R}$ to $\tilde{\mathbb{R}}$.
    \item
      Show that $f$ is smooth as a map from $\tilde{\mathbb{R}}$ to $\mathbb{R}$ if and only if $f^{(n)}(0) = 0$ whenever $n$ is not an integral multiple of 3.
  \end{enumerate}
\end{customprob}

\begin{proof}
  $ $
  \begin{enumerate}[label=(\alph*)]
    \item 
      The ``$\psi \circ f \circ \phi^{-1}$" is simply $f^3$, which is a smooth map from $\mathbb{R}$ to $\mathbb{R}$.
      Thus $f: \tilde{\mathbb{R}} \rightarrow \mathbb{R}$ is smooth.
    \item
      \todo[inline,caption={}]{
        Solve this!
      }
  \end{enumerate}
\end{proof}

\begin{customprob}{2-6}
  Let $P: \mathbb{R}^{n + 1} \setminus \{ 0 \} \rightarrow \mathbb{R}^{k + 1} \setminus \{ 0 \}$ be a smooth function, and suppose that for some $d \in \mathbb{Z}$, $P(\lambda x) = \lambda^dP(x)$ for all $\lambda \in \mathbb{R} \setminus \{ 0 \}$ and $x \in \mathbb{R}^{n + 1} \setminus \{ 0 \}$.
  Show that the map $\tilde{P}: \RP^n \rightarrow \RP^k$ defined by $\tilde{P}([x]) = [P(x)]$ is well defined and smooth.
\end{customprob}

\begin{proof}
  Let $P_1, \cdots, P_{k + 1}$ denote the component functions of $P$.

  Suppose $[x_1:\cdots:x_{n + 1}] = [y_1:\cdots:y_{n + 1}]$.
  Then there exists $\lambda \ne 0$ such that $(y_1, \cdots, y_{n + 1}) = (\lambda x_1, \cdots, \lambda x_{n + 1})$.
  $P(y_1, \cdots, y_{n + 1}) = P(\lambda x_1, \cdots, \lambda x_{n + 1}) = \lambda^dP(x_1, \cdots, x_{n + 1})$.
  Since $\lambda^d \ne 0$, $[P(y_1, \cdots, y_{n + 1})] = [P(x_1, \cdots, x_{n + 1})]$.
  Therefore, $\tilde{P}$ is well-defined.

  Let $\tilde{p} = [p_1:\cdots:p_{n + 1}] \in \RP^n$ be given.
  Without loss of generality, assume $p_{n + 1} \ne 0$.
  Consider the chart $(U, \psi_{n + 1})$ with $U = \{ [x_1:\cdots:x_{n + 1}] \mid x_{n + 1} \ne 0 \}$.
  Let $q_i = P_i(p_1, \cdots, p_{n + 1})$.
  Without loss of generality, assume $q_{k + 1} \ne 0$.
  Then $\tilde{P}(\tilde{p})$ is contained in $V = \{ [y_1:\cdots:y_{k + 1}] \mid y_{k + 1} \ne 0 \}$.
  Since $P$ is smooth, there exists $0 < \delta < \abs{x_{n + 1}}$ such that $\abs{(x_1, \cdots, x_{n + 1}) - (p_1, \cdots, p_{n + 1})} < \delta$ implies $P_{k + 1}(x_1, \cdots, x_{n + 1}) \ne 0$.
  Then $[p_1:\cdots:p_{n+1}] \in \pi(B(p_1, \cdots, p_{n + 1})) \subset U \cap F^{-1}(V)$.
  Therefore, $U \cap F^{-1}(V)$ is open in $\RP^n$.

  Finally the composition map $\psi_{k + 1} \cdot \tilde{P} \cdot \phi_{n + 1}^{-1}$ sends $(x_1 / x_{n + 1}, \cdots, x_n / x_{n + 1})$ to $(y_1 / y_{k + 1}, \cdots, y_k / y_{k + 1})$ where $y_i = P_i(x_1, \cdots, x_{n + 1})$.
  In other words, $(x_1, \cdots, x_n) \mapsto (y_1 / y_{k + 1}, \cdots, y_k / y_{k + 1})$ where $y_i = P_i(x_1, \cdots, x_n, 1)$.
  Since each $P_i$ is smooth, this map must be smooth as well.
  By (\ref{proof_prop_2_5}), $\tilde{P}$ is smooth.
\end{proof}

\begin{customprob}{2-7}
  Let $M$ be a nonempty smooth $n$-manifold with or without boundary, and suppose $n \geq 1$.
  Show that the vector space $C^{\infty}(M)$ is infinite-dimensional.
\end{customprob}

\begin{proof}
  Let $k \in \mathbb{N}$ be given.
  Let $p \in M$ be chosen arbitrarily.
  Let $(U, \phi)$ be a smooth chart containing $p$.
  Then $\hat{U} = \phi(U)$ is an open subset of $\mathbb{R}^n$ or $\mathbb{H}^n$.
  In each case, we can pick $k$ distinct points $x_1, \cdots, x_k \in \hat{U}$ because $\hat{U}$ is a nonempty open subset and $n \geq 1$.
  Since $\hat{U}$ is open, there exist open $U_1, \cdots, U_k$ such that $x_i \in U_i \subset \hat{U}$ and $U_i \cap U_j$ whenever $i \ne j$.
  Moreover, $\{ x_i \}$ is a closed subset.
  By Proposition 2.25, we obtain $k$ bump functions $f_i$ for $\{ x_i \}$ supported in $U_i$.
  Extend each $f_i$ by setting $f_i(q) = 0$ for any $q \notin U$.
  Then each $f_i$ lives in $C^{\infty}(M)$.
  Clearly, $\sum c_if_i = 0$ implies $c_i = 0$, so $\{ f_1, \cdots, f_k \}$ is linearly independent.
  Therefore, $C^{\infty}(M)$ is infinite-dimensional.
\end{proof}

\begin{customprob}{2-14}
  Suppose $A$ and $B$ are disjoint closed subsets of a smooth manifold $M$.
  Show that there exists $f \in C^{\infty}(M)$ such that $0 \leq f(x) \leq 1$ for all $x \in M$, $f^{-1}(0) = A$ and $f^{-1}(1) = B$.
\end{customprob}

\begin{proof}
  By Theorem 2.29, there exist $\alpha, \beta \in C^{\infty}(M)$ such that $\alpha^{-1}(0) = A$ and $\beta^{-1}(0) = B$.
  Then $f(x) = \alpha(x) / (\alpha(x) + \beta(x))$ is a desired map.
\end{proof}
