\begin{customexer}{2.1}
  Let $M$ be a smooth manifold with or without boundary.
  Show that pointwise multiplication turns $C^{\infty}(M)$ into a commutative ring and a commutative and associative algebra over $\mathbb{R}$.
\end{customexer}

\begin{proof}
  $ $
  \begin{itemize}
    \item
      The constant map $f(p) = 0$ is clearly in $C^{\infty}(M)$ and it is the additive identity.
    \item
      The constant map $f(p) = 1$ is clearly in $C^{\infty}(M)$ and it is the multiplicative identity.
    \item
      Let $f \in C^{\infty}(M), g \in C^{\infty}(M)$.
      Let $p \in M$ and $(\phi, U)$ be a smooth chart for $p$.
      Then $f \circ \phi^{-1}$ and $g \circ \phi^{-1}$ are both smooth, real-valued maps defined on an open subset of $\mathbb{R}^n$.
      Thus $f + g$ is in $C^{\infty}(M)$ 
      Moreover, $f + g = g + f$ because addition in $\mathbb{R}$ is commutative.
    \item
      Let $f, g, h \in C^{\infty}(M)$.
      Let $p \in M$ and $(\phi, U)$ be a smooth chart for $p$.
      Then $f \circ \phi^{-1}$ and $g \circ \phi^{-1}$ are both smooth, real-valued maps defined on an open subset of $\mathbb{R}^n$.
      Therefore, $fg$ is in $C^{\infty}(M)$
      Moreover, $fg = gf$ and $(fg)h = f(gh)$ because multiplication in $\mathbb{R}$ is commutative and associative.
    \item
      Let $c \in \mathbb{R}, f \in C^{\infty}(M)$.
      Then $cf$ can be seen as $fg$ where $g$ is the constant function whose value is $c$.
      As shown above, $cf \in C^{\infty}(M)$.
  \end{itemize}
\end{proof}

\begin{customexer}{2.2}
  Let $U$ be an open submanifold of $\mathbb{R}^n$ with its standard smooth manifold structure.
  Show that a function $f: U \rightarrow \mathbb{R}^k$ is smooth in the sense just defined if and only if it is smooth in the sense of ordinary calculus.
  Do the same for an open submanifold with boundary in $\mathbb{H}^n$.
\end{customexer}

\begin{proof}
  $f$ is smooth in the sense just defined if and only if $f^{-1} \circ \Id^{-1}$ is smooth in the sense of ordinary calculus.
  Since $f^{-1} \circ \Id^{-1} = f^{-1}$, $f^{-1} \circ \Id^{-1}$ is smooth in the sense of ordinary calculus if and only if $f^{-1}$ is smooth in the sense of ordinary calculus.
\end{proof}
