\begin{customthm}{4.1}
  Suppose $F: M \rightarrow N$ is a smooth map and $p \in M$.
  If $dF_p$ is surjective, then $p$ has a neighborhood $U$ such that $F\vert_U$ is a submersion.
  If $dF_p$ is injective, then $p$ has a neighborhood $U$ such that $F\vert_U$ is an immersion.
\end{customthm}

\begin{proof}
  Let $(U, \phi)$ be a chart containing $p$ and $(V, \psi)$ be a chart containing $F(p)$.
  We may assume $F(U) \subset V$.
  It suffices to show that if the Jacobian of $F$ with respect to $(U, \phi)$ is full rank at $p$, then it is full rank in some neighborhood of $p$ contained in $U$.
  Example 1.28 in the textbook shows that the set of full rank matrices is an open subset of $M(m \times n, \mathbb{R})$.
  We will use the notation $J\vert_{q}$ to denote the Jacobian of $F$ with respect to $(U, \phi)$ at $q \in U$.
  Then $J\vert_{p}$ is an element of an open subset of $M(m \times n, \mathbb{R})$.
  Each entry of $J\vert_{q}$ is of the from $\frac{\partial}{\partial x^i}(\psi^j \circ F \circ \phi)(\phi(q))$ where each $(\frac{\partial}{\partial x^i}(\psi^j \circ F \circ \phi)) \circ \phi$ is a smooth function.
  Therefore, there exists a neighborhood of $p$ such that the Jacobian matrix of $F$ with respect to $(U, \phi)$ is full rank.
\end{proof}

\begin{customexer}{4.3(Verification of Example 4.2)}
  Verify the following claims:
  \begin{enumerate}[label=(\alph*)]
    \item
      Suppose $M_1, \cdots, M_k$ are smooth manifolds.
      Then each of the projection maps $\pi_i: M_1 \times \cdots \times M_k \rightarrow M_i$ is a smooth submersion.
    \item
      If $\gamma: J \rightarrow M$ is a smooth curve in a smooth manifold $M$ with or without boundary, then $\gamma$ is a smooth immersion if and only if $\gamma'(t) \ne 0$ for all $t \in J$.
  \end{enumerate}
\end{customexer}

\begin{proof}
   $ $
  \begin{enumerate}[label=(\alph*)]
    \item
      Let $d_1, \cdots, d_k$ denote the dimensions of $M_1, \cdots, M_k$, respectively.
      Let $M = M_1 \times \cdots \times M_k$.

      (\ref{problem_2_2}) implies that $\pi_i$ is smooth for each $i$ by setting $F = \Id: M \rightarrow M$.
      Let $p = (p_1, \cdots, p_k) \in M$.
      Thus it suffices to show that the dimension of $d(\pi_i)_p(T_p(M))$ is the same as the dimension of $T_{p_i}(M_i)$.

      By Proposition 3.12, $\dim(T_p(M)) = \sum d_i$.
      Since the $\alpha$ defined in (\ref{problem_3_2}) is an isomorphism,
      \begin{equation}\label{exercise_4_3_eq_1}
        \dim(d(\pi_1)_p(T_p(M)) \oplus \cdots \oplus d(\pi_k)_p(T_p(M))) = \dim(T_p(M)) = \sum d_i.
      \end{equation}

      However, for each $i$, $d(\pi_i)_p(T_p(M)) \subset T_{p_i}M_i$.
      Thus $\dim(d(\pi_i)_p(T_p(M))) \leq \dim(T_{p_i}M_i) = d_i$.
      By (\ref{exercise_4_3_eq_1}), $\dim(d(\pi_i)_p(T_p(M))) = \dim(T_{p_i}M_i)$.
    \item
      $\gamma$ is a smooth immersion if and only if $d\gamma_t: T_tJ \rightarrow T_{\gamma(t)}M$ is injective for each $t \in J$.
      Since each $T_tJ$ is a 1-dimensional vector space spanned by $d/dt\vert_t$, $d\gamma_t$ is injective if and only if $d\gamma_t$ sends the basis element to a nonzero element.
      Finally, $\gamma'(t) = d\gamma(d/dt\vert_{t})$.
      Therefore, $\gamma$ is a smooth immersion if and only if $\gamma'(t) \ne 0$ for all $t \in J$.
  \end{enumerate}
\end{proof}

\begin{customexer}{4.4}
  Show that a composition of smooth submersions is a smooth submersion, and a composition of smooth immersions is a smooth immersion.
  Give a counterexample to show that a composition of maps of constant rank need not have constant rank.
\end{customexer}

\begin{proof}
  Let $M, N, L$ be smooth manifolds with or without boundary, and $F: M \rightarrow N, G: N \rightarrow L$ be given.
  If $F, G$ are submersions, $dF_p$ and $dG_{F(p)}$ are surjective for each $p$.
  Then $d(G \circ F)_p = dG_{F(p)} \circ dF_p$ is surjective for each $p$ by (\ref{proof_prop_3_6}).
  Thus a composition of smooth submersions is a smooth submersion.
  By the exact same argument, a composition of smooth immersions is a smooth immersion.
  \todo[inline,caption={}]{
    Counterexample?
  }
\end{proof}
