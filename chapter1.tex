\begin{customexer}{1.1}\label{exercise_1_1}
  Show that equivalent definitions of manifolds are obtained if instead of allowing $U$ to be homeomorphic to \textit{any} open subset of $\mathbb{R}^n$, we require it to be homeomorphic to an open ball in $\mathbb{R}^n$, or to $\mathbb{R}^n$ itself.
\end{customexer}

\begin{proof}
  It is clear that a ``manifold" satisfying the open-ball or $\mathbb{R}^n$ definition satisfies the open-subset definition.
  Let $M$ be a manifold satisfying the open-subset definition.
  Let $x \in M$ be given and let $U, \hat{U}, \phi$ be given according to the definition.
  Since $\hat{U}$ is open, there exists an open ball $B$ such that $\phi(x) \in B \subset \hat{U}$.
  Restrict $\phi$ to $\phi^{-1}(B)$.
  Then $\phi^{-1}(B)$ is an open subset of $M$ containing $x$, and $\phi\mid_{\phi^{-1}(B)}$ is a homeomorphism between $\phi^{-1}(B)$ and $B$.
  Thus $M$ satisfies the open-ball definition.

  $B(x, r) \subset \mathbb{R}^n$ is homeomorphic to $\mathbb{R}^n$ by the map $(x_1 + a_1, \cdots, x_n + a_n) \mapsto (\frac{a_1}{r - a_1}, \cdots, \frac{a_n}{r - a_n})$ where $x = (x_1, \cdots, x_n)$ is the center of $B(x, r)$ and $r$ is the radius.
  Since the composition of two homeomorphisms gives a homeomorphism, $M$ also satisfies the $\mathbb{R}^n$ definition as well.
\end{proof}


\begin{customexer}{1.6}
  Show that $\RP^n$ is Hausdorff and second-countable, and is therefore a topological $n$-manifold.
\end{customexer}

\begin{proof}
  From the definition of $\pi$, it is easy to see that $\pi(B(x, r))$ is open in $\RP^n$ where $x \in S^n$ and $0 < r < 1$.

  Let $[x], [y] \in \RP^n$ be given.
  Without loss of generality, assume $x, y \in S^{n}$.
  Let $r = \min\{ \abs{x - y}, \abs{x + y}, 1 \} / 2$.
  Then $U_x = \pi(B(x, r)), U_y = \pi(B(y, r))$ contain $[x], [y]$, respectively.
  $\pi^{-1}(U_x), \pi^{-1}(U_y)$ are both open in $\mathbb{R}^{n + 1} \setminus \{ 0 \}$ which can be seen easily by writing down exactly which points belong to them, so $U_x, U_y$ are both open in $\RP^n$.
  Then $\pi^{-1}(U_x \cap U_y) = \pi^{-1}(U_x) \cap \pi^{-1}(U_y) = \emptyset$, so $U_x \cap U_y = \emptyset$.
  Therefore, $\RP^n$ is Hausdorff.

  Let $\mathcal{B} = \{ \pi(B(x, 1 / k)) \mid x \in \mathbb{Q}^{n + 1} \cap S^{n}, k \in \{ 2, 3, 4, \cdots \} \}$.
  Then $\mathcal{B}$ is a countable collection of open sets whose union is $\RP^n$.
  Let $U \subset \RP^n$ be a nonempty open set.
  Let $[x] \in U$.
  Since $\pi$ is a quotient map, $\pi^{-1}(U)$ is open.
  Moreover, $x \in \pi^{-1}(U)$.
  Without loss of generality, $x \in S^{n}$.
  Then $x \in B(x', 1 / k) \subset \pi^{-1}(U)$ for some $B(x', 1 / k) \in \mathcal{B}$.
  Then $[x] = \pi(x) \in \pi(B(x', 1 / k)) \subset \pi(\pi^{-1}(U)) = U$.
  Therefore, $\mathcal{B}$ is a countable basis of $\RP^n$.
\end{proof}

\begin{customexer}{1.7}
  Show that $\RP^n$ is compact.
\end{customexer}

\begin{proof}
  $\pi(S^n) = \RP^n$ and $S^n$ is compact because it is a closed, bounded subset of $\mathbb{R}^{n + 1}$. (Heine-Borel)
  Moreover, the image of a compact set under a continuous map is compact. (See A.45(a))
  Thus $\RP^n$ is compact.
\end{proof}

\begin{customexer}{1.14}
  Suppose $\mathcal{X}$ is a locally finite collection of subsets of a topological space $M$.
  \begin{enumerate}[label=(\alph*)]
    \item 
      The collection $\{ \overline{X} : X \in \mathcal{X} \}$ is also locally finite.
    \item
      $\overline{\bigcup_{X \in \mathcal{X}} X} = \bigcup_{X \in \mathcal{X}} \overline{X}$.
  \end{enumerate}
\end{customexer}

\begin{proof}
  $ $
  \begin{enumerate}[label=(\alph*)]
    \item
      Let $p \in M$.
      Then there exists an open set $U$ containing $x$ such that there are only finitely many $X \in \mathcal{X}$ such that $U \cap X \ne \emptyset$.
      Let $X \in \mathcal{X}$.
      \begin{itemize}
        \item
          If $U \cap X \ne \emptyset$, then $U \cap \overline{X} \supset U \cap X \ne \emptyset$.
        \item
          If $U \cap X = \emptyset$, then $U^c$ is closed, so $\overline{X} \subset U^c$.
          In other words, $U \cap \overline{X} = \emptyset$.
      \end{itemize}
      This shows that the number of $X \in \mathcal{X}$ that intersects $U$ and the number of $\overline{X} \in \mathcal{X}$ that intersects $U$ are the same.
      Therefore, $\{ \overline{X} : X \in \mathcal{X} \}$ is also locally finite.
    \item
      Since the closure of a set is defined to be the intersection of all closed sets containing it, $\bigcup_{X \in \mathcal{X}} \overline{X} \subset \overline{\bigcup_{X \in \mathcal{X}} X}$.
      Let $x \notin \bigcup_{X \in \mathcal{X}} \overline{X}$.
      Then there exists a neighborhood $U$ of $x$ such that $U$ intersects only finitely many $X \in \mathcal{X}$.
      Let $X_1, \cdots, X_n$ denote them.
      By the same argument as part (a), $\overline{X_1}, \cdots, \overline{X_n}$ are the only elements in $\{ \overline{X} \mid X \in \mathcal{X} \}$ that $U$ intersects.
      Since $x \notin \overline{X_i}$ for each $i = 1, \cdots, n$, $U^c \cup \overline{X_1} \cup \cdots \cup \overline{X_n}$ is a closed set which contains all $X \in \mathcal{X}$ but does not contain $x$.
      In other words, $x \notin \overline{\bigcup_{X \in \mathcal{X}} X}$.
  \end{enumerate}
\end{proof}

\begin{customexer}{1.18}
  Let $M$ be a topological manifold.
  Two smooth atlases for $M$ determine the same smooth structure if and only if their union is a smooth atlas.
\end{customexer}

\begin{proof}
  Let $\mathcal{A}, \mathcal{A}'$ be two smooth atlases.

  Suppose that they determine the same smooth structure $\mathcal{B}$.
  Then $\mathcal{A} \cup \mathcal{A}' \subset \mathcal{B}$, so $\mathcal{A} \cup \mathcal{A}'$ must be a smooth atlas.
  By Proposition 1.17(a), $\mathcal{A} \cup \mathcal{A}'$ determines a unique smooth structure, but it must be $\mathcal{B}$ because $\mathcal{B}$ contains the union.

  On the other hand, suppose that their union is a smooth atlas.
  Let $\mathcal{B}$ be the smooth structure that the union determines.
  Such $\mathcal{B}$ must exist by Proposition 1.17(a).
  By the same proposition, $\mathcal{A}, \mathcal{A}'$ must determine the unique smooth structures.
  However, they must be $\mathcal{B}$ because $\mathcal{B}$ contains both $\mathcal{A}$ and $\mathcal{A}'$.
\end{proof}

\begin{customexer}{1.20}
  Every smooth manifold has a countable basis of regular coordinate balls.
\end{customexer}

\begin{proof}
  Let $M$ be an $n$-dimensional smooth manifold.
  We consider the special case that there exists a single chart $(\phi, U)$ with $U = M$.
  Let $x \in \hat{U}$ with rational coordinates.
  Then there exists $s > 0$ such that $B(x, s) \subset \hat{U}$.
  For each rational number $r \in (0, s)$, we consider the chart $(p \mapsto \phi(p) - x, \phi^{-1}(B(x, r)))$.

  Let $\mathcal{B}$ be the smooth atlas consisting of all such charts for each $x \in \hat{U}$ and $r$.
  \begin{itemize}
    \item
      $\mathcal{B}$ is a countable collection because $x \in \mathbb{Q}^n$ and $r \in \mathbb{Q}$.
    \item
      Let $(p \mapsto \phi(p) - x, \phi^{-1}(B(x, r))) \in \mathcal{B}$ be given.
      Then there exists a chart $(p \mapsto \phi(p) - x, \phi^{-1}(B(x, r')))$ in $\mathcal{B}$ with $r' > r$.
      Let $B = \phi^{-1}(B(x, r)), B' = \phi^{-1}(B(x, r'))$.
      Let $\psi$ denote the map $p \mapsto \phi(p) - x$.
      Then $\psi(B) = B(0, r)$ and $\psi(B') = B(0, r')$, respectively.
  \end{itemize}
  \todo[inline,caption={}]{
    Finish this proof!
  }
\end{proof}

\begin{customexer}{1.39}
  Let $M$ be a topological $n$-manifold with boundary.
  \begin{enumerate}[label=(\alph*)]
    \item
      $\Int M$ is an open subset of $M$ and a topological $n$-manifold without boundary.
    \item
      $\partial M$ is a closed subset of $M$ and a topological $(n - 1)$-manifold without boundary.
    \item
      $M$ is a topological manifold if and only if $\partial M = \emptyset$.
    \item
      If $n = 0$, then $\partial M = \emptyset$ and $M$ is a 0-manifold.
  \end{enumerate}
\end{customexer}

\begin{proof}
  $ $
  \begin{enumerate}[label=(\alph*)]
    \item
      Let $x \in \Int M$.
      Let $(\phi, U)$ be an interior chart for $x$.
      Then $x \in U \subset \Int M$ because every point in $U$ is in an interior chart $(\phi, U)$.
      A subspace of $M$ must be Hausdorff and second-countable by Proposition A.17(g, i), so $\Int M$ is a second-countable, Hausdorff space in which every point has a neighborhood homeomorphic to an open subset in $\mathbb{R}^n$.
      Thus $\Int M$ is an $n$-manifold without boundary.
    \item
      Since $\partial M = M \setminus \Int M$ and $\Int M$ is open in $M$, $\partial M$ is closed in $M$.
      Let $x \in \partial M$.
      Let $(\phi, U)$ be a boundary chart of $x$.
      If a point $y \in U$ gets mapped into $\Int \mathbb{H}^n$, then it is certainly an interior point.
      Thus $\phi(U \cap \partial M) \subset \partial \mathbb{H}^n$.
      Then $\pi_{n - 1} \circ \phi$ is a homeomorphism that maps $U \cap \partial M$ into an open subset of $\mathbb{R}^{n - 1}$ where $\pi_{n - 1}: (x_1, \cdots, x_n) \mapsto (x_1, \cdots, x_{n - 1 })$.
    \item
      If $\partial M$ is empty, then $M = \Int M$, so (a) implies that $M$ is an $n$-dimensional manifold.
      If $M$ is a topological manifold, every point is an interior point.
      Since a point cannot be both an interior point and a boundary point, $\partial M$ is empty.
    \item
      If $n = 0$, then $\partial \mathbb{H}^0 = \emptyset$.
      Thus, the condition that $\phi(U) \cap \partial \mathbb{H}^n \ne \emptyset$ can never be satisfied, so there cannot be any boundary point.
  \end{enumerate}
\end{proof}
