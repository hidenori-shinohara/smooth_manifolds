\begin{customexer}{1.1}
  Show that equivalent definitions of manifolds are obtained if instead of allowing $U$ to be homeomorphic to \textit{any} open subset of $\mathbb{R}^n$, we require it to be homeomorphic to an open ball in $\mathbb{R}^n$, or to $\mathbb{R}^n$ itself.
\end{customexer}

\begin{proof}
  It is clear that a ``manifold" satisfying the open-ball or $\mathbb{R}^n$ definition satisfies the open-subset definition.
  Let $M$ be a manifold satisfying the open-subset definition.
  Let $x \in M$ be given and let $U, \hat{U}, \phi$ be given according to the definition.
  Since $\hat{U}$ is open, there exists an open ball $B$ such that $\phi(x) \in B \subset \hat{U}$.
  Restrict $\phi$ to $\phi^{-1}(B)$.
  Then $\phi^{-1}(B)$ is an open subset of $M$ containing $x$, and $\phi\mid_{\phi^{-1}(B)}$ is a homeomorphism between $\phi^{-1}(B)$ and $B$.
  Thus $M$ satisfies the open-ball definition.

  $B(x, r) \subset \mathbb{R}^n$ is homeomorphic to $\mathbb{R}^n$ by the map $(x_1 + a_1, \cdots, x_n + a_n) \mapsto (\frac{a_1}{r - a_1}, \cdots, \frac{a_n}{r - a_n})$ where $x = (x_1, \cdots, x_n)$ is the center of $B(x, r)$ and $r$ is the radius.
  Since the composition of two homeomorphisms gives a homeomorphism, $M$ also satisfies the $\mathbb{R}^n$ definition as well.
\end{proof}


\begin{customexer}{1.6}
  Show that $\RP^n$ is Hausdorff and second-countable, and is therefore a topological $n$-manifold.
\end{customexer}

\begin{proof}
  From the definition of $\pi$, it is easy to see that $\pi(B(x, r))$ is open in $\RP^n$ where $x \in S^n$ and $0 < r < 1$.

  Let $[x], [y] \in \RP^n$ be given.
  Without loss of generality, assume $x, y \in S^{n}$.
  Let $r = \min\{ \abs{x - y}, \abs{x + y}, 1 \} / 2$.
  Then $U_x = \pi(B(x, r)), U_y = \pi(B(y, r))$ contain $[x], [y]$, respectively.
  $\pi^{-1}(U_x), \pi^{-1}(U_y)$ are both open in $\mathbb{R}^{n + 1} \setminus \{ 0 \}$ which can be seen easily by writing down exactly which points belong to them, so $U_x, U_y$ are both open in $\RP^n$.
  Then $\pi^{-1}(U_x \cap U_y) = \pi^{-1}(U_x) \cap \pi^{-1}(U_y) = \emptyset$, so $U_x \cap U_y = \emptyset$.
  Therefore, $\RP^n$ is Hausdorff.

  Let $\mathcal{B} = \{ \pi(B(x, 1 / k)) \mid x \in \mathbb{Q}^{n + 1} \cap S^{n}, k \in \{ 2, 3, 4, \cdots \} \}$.
  Then $\mathcal{B}$ is a countable collection of open sets whose union is $\RP^n$.
  Let $U \subset \RP^n$ be a nonempty open set.
  Let $[x] \in U$.
  Since $\pi$ is a quotient map, $\pi^{-1}(U)$ is open.
  Moroever, $x \in \pi^{-1}(U)$.
  Without loss of generality, $x \in S^{n}$.
  Then $x \in B(x', 1 / k) \subset \pi^{-1}(U)$ for some $B(x', 1 / k) \in \mathcal{B}$.
  Then $[x] = \pi(x) \in \pi(B(x', 1 / k)) \subset \pi(\pi^{-1}(U)) = U$.
  Therefore, $\mathcal{B}$ is a countable basis of $\RP^n$.
\end{proof}

\begin{customexer}{1.7}
  Show that $\RP^n$ is compact.
\end{customexer}

\begin{proof}
  $\pi(S^n) = \RP^n$ and $S^n$ is compact because it is a closed, bounded subset of $\mathbb{R}^{n + 1}$. (Heine-Borel)
  Moreover, the image of a compact set under a continuous map is compact. (See A.45(a))
  Thus $\RP^n$ is compact.
\end{proof}
